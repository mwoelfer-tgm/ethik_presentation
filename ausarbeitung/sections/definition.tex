%!TEX root=../document.tex

\section{Allgemein}
\subsection{Definition}
\textit{\textbf{Als Zahlungsverfahren werden alle Formen und Prozesse der Übertragung von Eigentumsrechten an Zahlungsmitteln bezeichnet.}}

Andere Bezeichnungen: \textbf{Bezahlverfahren}, \textbf{Zahlungssysteme}, \textbf{Zahlungsinstrumente}.

\subsection{Klassifizierungen}
\subsubsection{Klassisch vs Elektronisch}
Oft wird in \textbf{elektronische} und \textbf{klassische} Zahlungssysteme unterschieden, beispielsweise stellen Zahlungsarten wie Nachnahme, Scheck oder Überweisung ein klassisches Zahlungssystem dar. Wichtig dabei ist, dass die Abrechnung \textbf{vor} oder \textbf{nach} der Bestellung bzw. Lieferung erfolgt. 

Bei elektronischen Zahlungssysteme erfolgt die Zahlung unmittelbar über elektronische Medien. Beispiele dafür wären Kreditkartenzahlungen, Bankomatzahlungen, Lastschriftzahlungen. 

Diese Unterscheidung weist jedoch einige Probleme auf, unter anderem ist das unabdingbare \textbf{Online-Banking} nicht eindeutig einer Kategorie zuzuweisen. Zwar funktioniert es über elektronische Medien, aber bei Verwendung muss zuerst eingeloggt werden, und alle Daten \textbf{manuell} eingegeben werden. 

Ein weiteres Problem tritt bei der \textit{pay before}, \textit{pay now} und \textit{pay after} Unterscheidung. Man stelle sich ein Szenario vor, in welchem eine Rechnung am ende des Monats abgerechnet wird, aber mit einer \textit{Prepaidkarte} bezahlt wird. Es ist nicht klar definiert ob es sich um \textit{pay after} oder \textit{pay before} handelt.

\subsubsection{Bundesamt für Sicherheit der Informationstechnik}
Das \textbf{Budesamt für Sicherheit der Informationstechnik} unterscheidet lediglich zwischen \textbf{originären} und \textbf{abgeleiteten} Zahlungsverfahren.

Originäre Zahlungsverfahren umfassen die \textbf{physische} Übertragung von Geld aber auch die \textbf{Überweisung} und \textbf{Lastschrift}. Es sind jene Verfahren, auf welche alle anderen aufbauen und bilden somit das ''Fundament'' der Zahlungsverfahren.

Abgeleitete Zahlungsverfahren sind Verfahren welche fast ausschließlich über den elektronischen Handel fungieren und greifen letzten Endes auf ein originäres Verfahren zurück um die Übertragung abzuschließen.

\clearpage
\subsection{Zahlungsmöglichkeiten}
\subsubsection{Kategorisierung nach Einsatzort}
Wenn Käufer und Verkäufer sich physisch treffen für die Bezahlung, wird der (Verkaufs-)Ort als \textbf{Point of Sale} bezeichnet.

Bei einer Bezahlung welche nicht physisch sondern über \textbf{Telefon}, \textbf{Brief} oder \textbf{Internet} abläuft, spricht man von einem \textbf{Fernabsatz}

\subsubsection{Kategorisierung nach Betragshöhe}
\textbf{Macropayment}: Ab ungefähr 5€

\textbf{Micropayment}: Ungefähr 0,05€ bis 5€

\textbf{Nanopayment} (auch Millipayment, Minipayment oder Picpayment genannt): Bis ungefähr 0,05€ 

\subsubsection{Kategorisierung nach Herkunft}
Je nachdem ob sich der Kunde im \textbf{Inland} oder \textbf{Ausland} befindet kommt es zu einem anderen Zahlungsszenario, weil sich beispielsweise Steuern oder Lieferkosten erhöhen.


\subsubsection{Kategorisierung nach Häufigkeit}
\textbf{Einmalig}: Kunde und Verkäufer wickeln ein einmaliges Geschäft ab

\textbf{Wiederkehrend}: Kunde und Verkäufer wickeln ein sich wiederholendes Geschäft ab
 
Bei einer \textbf{wiederkehrenden} Leistung ist der Kunde bereit, mehr Registrierungsaufwand auf sich zu nehmen um die nachfolgenden Geschäfte leichter abzuwickeln. Der Verkäufer zieht Vorteil daraus, dass der Kunde ein besonders Vertrauensverhalten aufzeigt bei einer wiederkehrenden Leistung. 

\subsection{Anforderungen der Zahlungspflichtigen}
\subsubsection{Sicherheit}
Die wichtigste Anforderung der Käufer an den Verkäufer. Damit sind organisatorische und rechtliche Regelungen gemeint, welche dazu dienen Schäden des Zahlungspflichtigen zu vermeiden. Dabei werden vor allem folgende Punkte betrachtet:

\textbf{Transaktionskontrolle}: Es muss sicher gestellt werden, dass eine Transaktion immer erfolgreich durchgeführt wird. Bei unerwarteten Störungen, muss ein \textbf{rollback} durchgeführt werden. Es darf keine unberechte Transaktion durchgeführt werden

\textbf{Authentifizierung}: Es muss so schwer wie möglich gemacht werden den Kunden zu imitieren und auf dessen Kosten eine Zahlung durchzuführen

\textbf{Sperrmöglichkeit}: Es muss die Möglichkeit geben ein Konto sperren zu lassen um Zahlungen zu verhindern selbst wenn die Authentifikation von einem Dritten durchbrochen wurde. Beispiel: Bankomatkarte verlieren

\textbf{Haftungsbetrag}: Wenn bereits Authentifizierung durchbrochen wurde, und die Sperrung noch nicht aktiviert wurde, gibt der Haftungsbetrag jenen Wert an, welcher vom Kunden gezahlt werden muss um die Schäden zu begleichen.

\subsubsection{Installations- bzw. Registrierungsaufwand}
Bei der Erstverwendung eines Zahlungssystem muss ein bestimmter Aufwand aufgewandt werden. Dies kann eine einfache Registrierung der Daten sein aber es können auch Kosten für Hardware/Software auftreten.

\subsubsection{Kosten}
In der Regel ist das durchführen einer Zahlung mit einem Zahlungssystem gratis. Jedoch können Kosten für den Kunden auftreten wenn der Verkäufer Anschaffungskosten tragen muss (z.B. Kartenlesegerät) oder periodisch wiederkehrende Verwaltungskosten (z.B. Kreditkarte).

\subsubsection{Akzeptanzstellen}
Eine weitere sehr wichtige Anforderung des Kunden ist die Anzahl der Stellen an welchen das Zahlungsverfahren angenommen wird. Aus der Sicht des Kunden ist es unvorteilhaft sich den Aufwand zu machen für ein Zahlungssystem anzumelden um dieses anschließend nirgends verwenden zu können.

\subsection{Anforderungen der Zahlungsempfänger}
\subsubsection{Sicherheit}
Natürlich muss beim Zahlungsempfänger auch wieder die Sicherheit gegeben sein. Besonders wichtig ist hierbei wieder die \textbf{Transaktionskontrolle}, welche sicherstellt das alle Zahlungen auch tatsächlich beim Zahlungsempfänger ankommen. 

\subsubsection{Akzeptanz der Kunden}
Da man sich für fast alle neuen Zahlungssysteme registrieren muss, ist eine gewisse Hürde gegeben welche von den Kunden überwunden werden muss. Durch die hohe und fast obligatorische Verwendung der Überweisung-, Lastschrift- und Kreditkartenverfahren haben diese staatlich geregelten Verfahren einen klaren Vorteil. Ohne Akzeptanz und Verwenden der Käufer wird kein neues Zahlungssystem implementiert.

\subsubsection{Kosten}
Kosten des Zahlungsempfängers sind grundsätzlich in 3 Bereiche einzuteilen:

\textbf{Anschaffungskosten}: Einmalige Kosten welche Auftreten beim anschaffen von Hardware/Software Komponenten

\textbf{Periodische Kosten}: Kosten welche unabhängig von der Anzahl der Geschäfte auftreten, beispielsweise Lizenzkosten oder Grundgebühren

\textbf{Kosten bei einer Zahlung}: Kosten welche jeweils bei einer Zahlung auftreten, beispielsweise Verwaltungs- oder Autorisierungskosten




