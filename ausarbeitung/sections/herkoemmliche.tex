%!TEX root=../document.tex

\subsection{Herkömmliche Zahlsysteme}
\subsubsection{Definition}
Als herkömmliche Zahlungsverfahren werden jene bezeichneit, welche \textbf{nicht} elektronisch erfolgen. Zu diesen zählen unter anderem:
\begin{itemize}
	\item Überweisung
	\item Barzahlung
	\item Nachnahme
	\item Papiersheck
\end{itemize}
Aber auch der Tauschhandel ist ein herkömmliches Zahlsystem, wird heutzutage praktisch so gut wie nie eingesetzt.

\subsubsection{Barzahlung}
Bargeld ist das am meisten Verwendete herkömmliche Zahlsystem weltweit und gilt in den meisten Ländern als ein gesetzliches Zahlungsmittel. Es wurde zuerst als einheitliches Tauschmittel eingesetzt, hält nun aber mehr Funktionen inne. Diese wären:
\begin{itemize}
	\item Transaktionsmotiv
	\item Vorsichtsmotiv
	\item Spekulationsmotiv
\end{itemize}
Dieses Bargeld wird von der Staatsbank oder Zentralbank einer Gemeinschaft von Staaten (z.B. EU) geschaffen. Dieses Geld wird dann durch Geschäftsbanken in Umlauf gebracht.
Da durch diesen Vorgang Währung erschaffen wird, ist dieses Zahlsystem essenziell für alle anderen Zahlsysteme.


Die Barzahlung erfolgt zwischen 2 Beteiligten. Hierbei bezahlt der eine ein Produkt oder eine Dienstleistung, die andere Person erhält somit einen zuvor fix vereinbarten Betrag in Bar.
Eine Barzahlung ist ein indirekter Tauschvorgang (Ware gegen Geld, Geld gegen Ware). Ein direkter Tauschvorgang wäre Gut gegen Gut.

\subsubsection{Überweisung}
Bei einer Überweisung, lässt der Zahlende mittels Weisung an sein kontoführendes Buchgeld zu Lasten seines Girokontos an das Institut des Empfängers Geld übertragen. Es gibt somit 4 Beteiligte in diesem Prozess:
\begin{itemize}
	\item Auftraggeber
	\item kontoführende Bank des Auftraggebers
	\item kontoführende Bank des Zahlungsempfängers
	\item Zahlungsempfänger
\end{itemize}
80\% aller bargeldlosen Zahlungen in Deutschland durch Nichtbanken im Jahr 2013 wurden durch Überweisungen abgewickelt. Dies spiegelt den hohen Grad an Vertrauen zu dieser Zahlungsart im Volk dar.

\subsubsection{Nachnahme}
Bei dieser Art des Bezahlens bestimmt der Absender den Betrag den der Empfänger zu zahlen hat. Der Empfänger zahlt den Betrag bei der Auslieferung beim Zusteller, oder in einer Filiale des Zusenders. Das Bezahlen in einer Filiale ist zeitlich befristet. Der Zusender leitet das erhaltene Geld dem Absender weiter.
Es werden also 3 Beteiligte in dieser Methode benötigt:
\begin{itemize}
	\item Absender/Auftraggeber
	\item Zusteller
	\item Empfänger
\end{itemize}
Bekannte Anbieter von Zustellern, die diese Bezahlart unterstützen sind:
\begin{itemize}
	\item Österreichische Post
	\item Hermes Logistig Gruppe
	\item UPS
	\item DHL
\end{itemize}