\section{Ethische Fragestellungen}
Mithilfe den Schritten der ethischen Urteilsfindung wird der ethische Sachverhalt sowie die ethische Fragestellung dargestellt.

\subsection{Microtransactions}
\subsubsection{Sachverhalt}
Es ist möglich in Anwendungen, besonders im Bereich der Unterhaltung, für einen geringen Betrag sogenannte In-App-Käufe zu tätigen, welche je nach Anwendung einen kleinen Vorteil gegenüber Anderen bieten. In Applikationen, welche gezielt für Kinder entwickelt werden, wird nicht auf Microtransactions verzichtet. Moderne Zahlungssysteme ermöglichen einen einfach Ablauf der Zahlung.

\subsubsection{Fragestellung}
Ist es als Entwickler/Unternehmen in Ordnung leicht abwickelbare In-App-Käufe in Applikation für Kinder zu ermöglichen?

\subsection{Anonymität}
\subsubsection{Sachverhalt}
Es ist möglich über das Internet mithilfe von Kryptowährungen komplett anonym Geschäfte abzuwickeln. Diese Zahlungen sind erst 2 Jahre nach der Transaktion nachverfolgbar. 

\subsubsection{Fragestellung}
Sollte man anonymisierte Kryptowährungen verbieten um illegalen Bestellungen nachgehen zu können?

\subsection{Unreguliertes Zahlungssystem}
\subsubsection{Sachverhalt}
Kryptowährungen bauen ein unabhängig und vom Finanzsystem unreguliertes Zahlungssystem auf, welches nicht den staatlichen Regeln entspricht. 

\subsubsection{Fragestellung}
Sollten unabhängige und unregulierte Zahlungssysteme verboten werden, weil sie dem Finanzsystem schaden?

\section{Prüfungsfragen}

\begin{enumerate}
	\item Erläutern Sie die Schritte bei einer Zahlungsabwicklung per E-Payment
	\item Erklären Sie die ''Buchhaltung'' von Kryptowährungen
\end{enumerate}

